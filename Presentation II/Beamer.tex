\documentclass[10pt]{beamer}
\usetheme{jambro}
\graphicspath{{./graphics/}}

% Authors/institutes/date/etc
%--------------------------------
\title[]{Way Down in the Hole: Adaptation to Long-Term Water Loss in Rural India}
\author[{\scriptsize Roy Kisluk}]{\scriptsize Roy Kisluk}
\institute[{\scriptsize Under the supervision of Dr.\ Tomer Ifergane}]{\scriptsize Under the supervision of Dr.\ Tomer Ifergane}
\date{\scriptsize \today}

\begin{document}
%===================================================

\begin{frame}[plain]
	\titlepage{
		\begin{center}
			\begin{minipage}{0.8\textwidth}
				\centering
				\color{title!80}{\scriptsize Tel Aviv University\\
					Eitan Berglas School of Economics\\
				}
			\end{minipage}
		\end{center}}
\end{frame}


%===================================================
\section{Introduction}
\begin{frame}
	{Introduction}
	\begin{itemize}
		\item Israeli Arabs and ultra-Orthodox Jews have distinct consumption patterns
		\item Why is it important?
		      \begin{itemize}
			      \item Price increases may affect them differently \(\Rightarrow\)
			      \item Some groups may be more vulnerable to inflation of certain goods and services
		      \end{itemize}
		\item Methodology:
		      \begin{itemize}
			      \item Use CES and CPI data to estimate group expenditure shares
			      \item Estimate price index for each group
		      \end{itemize}
		\item Does inflation impact different demographic groups in Israel unequally?
	\end{itemize}
\end{frame}
%===================================================
\begin{frame}
	{Data and Methodology}
	\begin{itemize}
		\item Using household-level and individual level data from the Israeli Central Bureau of Statistics, we group individuals into three groups:
		      \begin{itemize}
			      \item Ultra-Orthodox
			      \item Israeli Arabs
			      \item Jews (excluding ultra-Orthodox)
		      \end{itemize}
		\item For each of these groups, we estimate the expenditure share of each good in their consumption basket
		\item Using prices data, we estimate the price index for each group
		\item We then compare the price indexes of these groups, along with the overall price index
		\item In progress:
	\end{itemize}
\end{frame}
%===================================================
\begin{frame}
	{Results}
	\begin{itemize}
		\item Different inflation for Arabs, Jews, and ultra-Orthodox
		\item These differences are driven by differences in consumption patterns
		\item In fact, they are more general, and can be grouped into food, goods, and services
		\item Potential results
	\end{itemize}
\end{frame}
%===================================================

\begin{frame}
	{Literature Review}
	\begin{itemize}
		\item Innovation benefits high-income consumers more; inflation heterogeneity across income.
		      \begin{itemize}
			      \item Jaravel (2021), Jaravel (2016)
		      \end{itemize}

		\item Trade shifts relative prices, reallocates labor across sectors.
		      \begin{itemize}
			      \item Cravino, Levchenko, and Rojas (2022)
		      \end{itemize}

		\item Income-driven demand + sectoral productivity explain service sector rise.
		      \begin{itemize}
			      \item Comin, Lashkari, and Mestieri (2022)
		      \end{itemize}

		\item Preferences, technology, prices explain structural transformation.
		      \begin{itemize}
			      \item Herrendorf, Rogerson, and Valentinyi (2013)
		      \end{itemize}

		\item Non-homothetic preferences explain sectoral shifts in spending and labor.
		      \begin{itemize}
			      \item Boppart (2014)
		      \end{itemize}

		\item Rising income shifts spending from goods to services.
		      \begin{itemize}
			      \item Aguiar and Bils (2015)
		      \end{itemize}
	\end{itemize}
\end{frame}

%===================================================
\begin{frame}
	{Overview}
	\begin{enumerate}
		\item Background
		\item Data
		\item Empirical Strategy
		\item Results
		\item Conclusion
		\item Thesis
	\end{enumerate}
\end{frame}

%===================================================
\section{Background}
\begin{frame}{Background}
	\begin{itemize}
		\item Israel exhibits significant consumption inequality among its main population groups: Arabs, Jews, and Ultra-Orthodox Jews.
		\item Such differences in consumption baskets lead to group-specific inflation experiences, resulting in inequality in unequal cost-of-living trends.
		\item Recent trends highlight persistent economic disparities; ultra-Orthodox and Arab households have lower average incomes compared to the general Jewish population.
		\item Therefore, they face higher poverty risks and financial vulnerabilities.
	\end{itemize}
\end{frame}
%===================================================
\begin{frame}{Background}
	\begin{itemize}
		\item In addition, demographic differences are central: Arabs and Ultra-Orthodox Jews have significantly higher natural population growth rates than non-Ultra-Orthodox Jews.
		\item Arabs and Ultra-orthodox consume more food and goods, and less services compared to the general Jewish population.
		\item Over time, these shifts are likely to reshape the structure of demand in the Israeli economy—placing growing weight on goods and food sectors relative to services.
		\item Understanding these trends is critical for anticipating structural transformation and assessing policy impacts on inequality and inflation.
		\item These patterns have significant implications for welfare, monetary policy transmission, and long-term economic development.
	\end{itemize}
\end{frame}
%===================================================
\section{Data}
\begin{frame}{Data}
	\begin{itemize}
		\item    Consumer Expenditure Survey (CES) - 2014-2022
		      \begin{itemize}
			      \item Public use files (PUF) provided by the CBS
			      \item Between 900 and 1400 products per year
			      \item Between 5000 and 9000 households per year
			      \item Indicators of the CBS to identify Arab and ultra-Orthodox households
			      \item Age group of the household head, income decile, number of household memebers, number of household earners, region of residence
			      \item Expenditure data (from the consumption survey): total expenditure and quantities, per product, per household.
		      \end{itemize}

		\item Consumer Price Index (CPI) - 2014-2022
		      \begin{itemize}
			      \item Public use files (PUF) provided by the Bank of Israel
			      \item Monthly prices for 38 categories of goods and services
			      \item Weights per category for verification and comparison
		      \end{itemize}
	\end{itemize}
\end{frame}
%===================================================
\section{Empirical Strategy}
\begin{frame}{Laspeyres Index}
	\[
		I_{t}=\frac{\sum_{j\in L}{\frac{P_{tj}}{P_{oj}}(P_{oj}Q_{oj})}}{\sum_{j\in L}P_{oj}Q_{oj}} \times 100
	\]
	Alternatively,
	\[
		I_{t}=\sum_{j\in L}W_{oj}I_{tj}\times 100
	\]
	\begin{itemize}


		\item Where: \( I_{tj} = \frac{P_{tj}}{P_{oj}}, \quad W_{oj} = \frac{P_{oj}Q_{oj}}{\sum_{j \in L} P_{oj}Q_{oj}} \)
		\item \( I_{t} \) - Index for period \( t \)
		\item \( Q_{oj} \) - Quantity of the good or service in the base period
		\item \( P_{oj} \) - Price of the good or service in the base period
		\item \( P_{tj} \) - Price of the good or service in period \( t \)
		\item \( L \) - The set of all goods and services in the index basket
	\end{itemize}
\end{frame}
%===================================================


\begin{frame}{Structural Model}
	The following utility function belongs to a subclass of Price Independent Generalized Linearity (PIGL) preferences (Muellbauer, 1975, 1976; Boppart, 2014; Cravino, Levchenko, and Rojas, 2022). \\
	The indirect utility of household \(h\) is:

	\[V^{h}\left(P_{t}^{s},P_{t}^{g},e_{t}^{h}\right)=\frac{1}{\epsilon}\left[\frac{e_{t}^{h}}{P_{t}^{s}}\right]^{\epsilon}-\frac{\nu_{t}^{h}}{\gamma}\left[\frac{P_{t}^{g}}{P_{t}^{s}}\right]^{\gamma}-\frac{1}{\epsilon}+\frac{\nu_{t}^{h}}{\gamma},\]
	Where:
	\begin{itemize}
		\item \(P_{t}^s\) and \(P^g_{t}\) are the prices of goods and services.
		\item \(0 \leq \varepsilon \leq \gamma \leq 1\) are parameters
		\item \(\nu_{t}^h \geq 0\) denotes household-specific age shifters
		      \begin{itemize}
			      \item \(\nu_{t}^h=\nu_{t}\mu^a\mu_{t}^h\), where \(\frac{1}{N_{t}}\sum_{h}\mu_{t}^h=1\)
			      \item The household-specific taste shifter has an aggregate component \( \nu_{t} \), an age-specific component \( \mu^a \), and an idiosyncratic component \( \mu_{t}^h \).
		      \end{itemize}
		\item \(e^t\) denotes expenditure.

	\end{itemize}
\end{frame}

\begin{frame}{Structural Model}
	\[
		\omega_t^{g, h} \equiv \frac{e_t^{g, h}}{e_t^h} = \nu_t^h
		\left[ \frac{P_t^s}{e_t^h} \right]^{\epsilon}
		\left[ \frac{P_t^{g}}{P_t^s} \right]^{\gamma},
	\]
\end{frame}

\begin{frame}{Structural Model}
	\[
		\Omega_t^g \equiv \frac{\sum_h e_t^{g,h}}{\sum_h e_t^h} =
		\left[ \frac{P_t^s}{e_t} \right]^c
		\left[ \frac{P_t^g}{P_t^s} \right]^\gamma
		\frac{1}{N_t} \sum_h \nu_t^h
		\left[ \frac{e_t^{h}}{e_t} \right]^{1 - c},
	\]
\end{frame}


\begin{frame}{Structural Model}
	\[
		\Omega_t^g =
		\left[ \frac{P_t^s}{e_t} \right]^c
		\left[ \frac{P_t^g}{P_t^s} \right]^\gamma
		\bar{\mu}_t \phi_t \nu_t.
	\]
\end{frame}

\begin{frame}{Structural Model}
	\[
		\ln \alpha_i^{y,h} = \beta_0 + \beta_1 \ln e_i^h + D^a + \delta_{r,t} + e_i^h,
	\]
\end{frame}

\begin{frame}{Structural Model}
	Price elasticity \(\gamma\) is estimated using the regression:

	\[
		\ln\Omega_{t}^{g}=b_{1}\ln P_{t}^{g}+b_{2}\ln P_{t}^{s}+b_{3}X_{t}+\ln\nu_{t},
	\]
	\begin{itemize}
		\item \(\Omega_{t}^g\) is the aggregate expenditure share on goods.
		\item \(X_{t}\equiv\ln\,(e_{t}^{-\epsilon}\bar{\mu}_{t}\phi_{t})\), where:
		      \begin{itemize}
			      \item \(e_{t}=\frac{1}{N_{t}}\sum_{h}e^h_{t}\) denotes average expenditures per household.
			      \item \({{\bar{\mu}_{t}\equiv\sum_{a}s_{t}^{a}\mu^{a}}}\) is the weighted average of the age-specific taste shifters, with weights given by expenditure shares \(s_{t}^a=\frac{e^a_{t}N_{t}^a}{e_{t}N_{t}}\), where \(a\) denotes that the variable is grouped by age group; \(\sum_{a}N_{t}^a=N_{t}\) is the total number of households in the economy.
			      \item \({{{\phi_{t}\equiv{\frac{1}{N_{t}}}\sum_{h}^{N_{t}}{\frac{\mu^{a}}{\bar{\mu}_{t}}}\left[{\frac{e_{t}^{h}}{e_{t}}}\right]}^{1-\epsilon}}}\) is a measure of the inequality in the economy, weighted by household preferences.
		      \end{itemize}
		\item \(b_{1}=\gamma\) (price elasticity).
		\item The other coefficients satisfy the restrictions \(b_{3} = 1\), and \(b_{2} = \varepsilon-b_{1}\).
		\item \(\nu_{t}\) is the aggregate taste shifter.
	\end{itemize}

\end{frame}

\end{document}


